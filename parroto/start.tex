\document-information[
	type = book
	type-params = "12pt;twoside"
	title= {Dummy Text}
	author= {Undefined Author}
]

\define[macro=anothervalue]{myname is}

\start-define[macro=complexname name="John" familyname="Smith" title="Dr."]
	complexname += macro(name="anothervalue")
	complexname += " "
	complexname += u"[{title} {familyname} {name}]".format(
			title = title, 
			familyname = familyname,
			name = name
		)
\stop

\start-style[apply-to=#zz]
    \style.font-weight{bold}
    \style.label{bold-label}
\stop

\start-style[apply-to=#xyz]
    \style.alignment{right}
\stop

\start-style[apply-to=#overlined]
	\style.text-line{overline}
	\style.line-height{common}
\stop

\document

\maketitle

\chapter[title="First chapter"]
Starting text - Part A.
\complexname[name="Juan" familyname="Pérez" title="Mr."]
\start-text[or="3",argument-1="44",argument-0="ABC"]
aaabbbé
\stop
\section[title="Section One" align=center]
\text[id=overlined]{Overlines}: Starting text - Part B: \complexname[name="Juan" familyname="Pérez"] and \complexname[name="Juan"]
\subsection[title="Subsection 1.1", label = "x"]
Starting text - Part C.
\section[title="Section Two"]
Starting text - Part D.
\text[id="zz" label={Inner example}]{aaa}
External \ref{x}part\ref[type=page]{x}x
\subsection[title="Subsection 2.1"]
Starting text - Part E.
\subsection[title="Subsection 2.2" id=xyz]
Starting text - Part F.
\subsection[title="Subsection 2.3"]
Starting text - Part G.
\section[title="Section Three"]
Starting text - Part H.
\chapter[title="Second chapter"]
Starting text - Part I.
\section[title="Section Second-One"]
Starting text - Part J.

